In conclusion, the methodology presented in this dissertation is a useful tool for transport planners to predict the trip attractiveness of a given area in Greater London based on its amenity and connectivity profile. By using SHAP values to extract local importance, we can identify destination hotspots based on the amenity profile of an area, which can be useful for local urban planners to identify areas that are popular for non-commute activities. However, the methodology is not without its limitations, and further work is needed to improve its generalisability and applicability in certain cases.

[to be completed]