% 650 words

Disentangling urban mobility patterns is a complex endeavour with a multitude of factors influencing destination and mode choice. Within that context, this work has sought to understand the relationship between the built environment of an area and the non-commute transport demand it attracts, using Greater London's public transport as a case study. We found that using a combination of amenity and connectivity features extracted from publicly available open data sources such as OpenStreetMap and the Census, we can predict the total number of arrivals as a proxy for trip attractiveness of a given area in London with a high degree of accuracy. The methodology calls for the use of an XGBoost model that has been fine-tuned to overcome overfitting and fairly evaluated using nested spatial k-fold cross-validation to avoid inflated performance due to spatial autocorrelation. 

In order to inspect feature importance on both the global and local levels in the prediction of the target variable made by the machine learning model, as opposed to a more conventional approach using a statistical model's parameters, we have employed SHAP explanation values to deconstruct the model's predictions. This has allowed us to identify the most important features in the model's decision-making process, as well as to understand the relationship between the features and the target variable. SHAP is a powerful explanation machine learning technique not only for highly complex deep learning models but also for tree-based models like XGBoost thanks to its interventional nature that can surface more sophisticated feature importance beyond XGBoost's built-in functionality that is tree-path dependent.

Following this methodology, we confirmed the high feature contribution of connectivity-related features such as the diversity index of transit routes in the area to the prediction of 'trip attractiveness' ---both globally and locally. However, we also found that amenity-related features such as the number of restaurants, parks, and theatres in the area are also important in predicting the target variable but with varying degrees of importance across the region of interest. This means the model and the accompanying local SHAP explanations, especially when presented cartographically, can highlight spatial heterogeneity in the impact of amenity to inflow travel demand, serving as a valuable tool for deducing other aspects of mobility, such as trip purposes. Finally, these SHAP values can be further manipulated to surface destination hotspots, which can be useful for transport urban planners to identify areas that are popular for non-commute activities. Specifically, in this study, where we focused on public transport, the predictive model can inform transit service planning and urban planning, such as in the case of redevelopment, as well as estimating pedestrian traffic surges, with applications in retail location planning and public health, among others. Since open data is at the core, the methodology can be extended to other cities and regions that lack more granular mobility datasets but have similar open data availability.

In conclusion, we have demonstrated the potential of using open data to predict non-commute transport demand in urban areas. The model's high accuracy, combined with the explanation power of SHAP, makes it a valuable tool for urban planners and transport authorities to understand the relationship between the built environment and trip attraction. Nevertheless, the work presented has limitations that could be addressed in future research. For instance, the model can be further improved by incorporating more granular data such as the frequency of public transport services, the quality of the built environment, and the socio-economic characteristics of the area. Moreover, the model can be extended to predict other types of transport demand, such as cycling and walking, which are also important modes of non-automobile intracity trips. Lastly, it hints at new possibilities in the field of urban mobility research that could be built upon. For example, we could also explore how this methodology and these insights could be incorporated into a more comprehensive mobility model using granular origin-destination matrices that enable the prediction of aspects of urban mobility, such as trip purposes, mode choices, accessibility, and activity sequence.




