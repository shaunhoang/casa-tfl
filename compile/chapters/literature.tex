% 2600 words

\subsection*{}

Understanding intracity non-commute travel behaviour on an aggregate scale is a crucial yet underresearched area in urban mobility planning, with more focus on commuting trips.

Estimating trip attraction constitutes an integral component of the demand forecast framework, termed the Four-Step Model (FSM), which begins with the Trip Generation step, where the number of trips generated by each traffic analysis zone is estimated, and Trip Distribution allocates these trips to their destination zones before proceeding to mode choice determination and route assignment. The task of Trip Distribution may be accomplished using a variety of spatial interaction models, most prominent among which is the classic \textit{Gravity Model} and its derivatives inspired by Newton's law of universal gravitation \citep{erlanderGravityModelTransportation1990}. Formulated initially to estimate flows among cities in a region, it has been adapted to intracity mobility, suggesting that the number of trips between two locations is proportional to each location's difference in \textit{mass} and inversely proportional to the distance between them. 

However, it is worth noting that the definition of \textit{mass} in any gravity-based spatial interaction model may depend on the research interest. The estimation of trip attraction in transport planning literature has predominantly revolved around commutes (i.e., primary travel demand) from residential zones to employment centres or education institutions. On the one hand, this choice is a practical one first and foremost, as commuting trips represent a large share of regular intracity mobility flows, whose high volume within short peak periods has important implications on road capacity management or public transit operations, among others. Due to its importance, the calibration of the spatial interaction model for commuting flows is further aided by nationwide flow datasets, such as those present in the UK Census \footnote{An interactive visualisation of the UK Census 2021 origin-destination data can be found at \url{https://www.ons.gov.uk/visualisations/censusorigindestination/}}, which provides detailed and explicit Origin-Destination data to serve as ground truth. Without this declared data, researchers have also turned to spatial features to calibrate commute trip distribution, as explored by \cite{yangLimitsPredictabilityCommuting2014}.

On the other hand, a plethora of potential trips generated for other non-commute purposes is often underresearched, partially due to the complexity of non-commute trip purpose identification, classification and quantification: Non-commute trips can be defined as trips that are not carried out to fulfil time-bound and regular work or education obligations and can include trips for shopping, leisure, and social activities. Therefore, as opposed to commuting trips, Non-commute trips are highly irregular both spatially and temporally, i.e., they can vary in terms of distance, duration, and mode of transport from one to another, even for the same destination between different days. Nevertheless, Non-commute trips are an important component of urban travel behaviour and can have a significant impact on the overall system. Advances in the field have brought about improvements to methodologies such as travel diary surveys\footnote{An example of these is the UK National Travel Survey, accessible at \url{https://www.gov.uk/government/collections/national-travel-survey-statistics}}, which collect data on personal and household travel behaviour, including Non-commute trips, to allow for comprehensive planning. By design, however, they are limited in scale.

Overall, this reveals a gap in our understanding of the comprehensive intracity travel behaviour on an aggregate scale. With the number of trips for shopping, leisure, and social activities increasing in recent years and the number of work-related trips decreasing as flexible work arrangements become more prevalent \citep{wohnerWorkFlexiblyTravel2022}, with recognised health benefits to workers \citep{macleodCommutingWorkPostpandemic2022}, it is crucial to understand the factors that influence non-commute trip attraction and to incorporate them into urban and transport planning models. Within the scope of this research, we hypothesise that a given area's non-commute trip attractiveness \textit{mass} is associated with the built environments, which include their amenity and public transport connectivity profiles. 

\section{Urban amenities as attractors}

\citet{cerveroTravelDemand3Ds1997} was one of the foundational works to provide an econometric estimation of how three groups of factors on the built environments---namely Density, Diversity and Design---are associated with travel demand to a traffic analysis zone for different purposes (work/non-work) and with different mode choices (private vehicles/public transport). This study is novel for demonstrating that changes in the built environment—such as density, diversity, and walkability have a tangible impact on travel behaviour, especially non-work one, which supports the idea that designing more compact, mixed-use, pedestrian-friendly neighbourhoods can significantly influence whereto and how people choose to travel. The components of density and diversity refer to the number and the variety of destinations available in a given area, respectively. In contrast, the design component refers to the physical layout of the area, including the presence of sidewalks, bike lanes, and other pedestrian-friendly infrastructure.

Many recent attempts have been to link the intensity of presence---or "Density"---of urban amenities to trip attraction and calibrate how we define urban attractors. Beyond the pure correlation observed between the density of points of interest (POIs) in a region and intensity of trips made to a given area \citep{melikovCharacterizingUrbanMobility2021}, the typology of the POIs differs between urban clusters that see distinct patterns of inflow trips made. For example, after \citet{aaqibjavedEstimationTripAttraction2020} have clustered urban areas in one city into Global (e.g., airports), Downtown (d.g., central business district) and Residential attractors based on the volume, spatial dispersion and distance of the trips made using origin-destination information extracted from call data, they found that the differences in the types of POIs in each of these clusters were also statistically significant. The typology of amenity POIs explored in this study included not only Retail, as is common when considering the primary purpose for non-commute trips, but also hospitals, public services, restaurants, religious institutions, and other categories. This suggests that the type of urban amenities available in an area can influence the volume of trips made to that area and that different types of amenities may attract different types of trip purposes.

Regarding "Diversity", \citet{cerveroTravelDemand3Ds1997} recognised the diversity of amenities is a factor that explains why people gravitate towards certain areas, apart from the presence of the amenities themselves. More specifically, for non-commute trips where locations are not necessarily prescribed or obligated, and the destination choice can be influenced by personal preferences and circumstantial convenience, people may be more likely to visit more vibrant areas that offer a variety of services and activities where multiple tasks could be carried out, as opposed to areas where one type of destinations is overrepresented. This subjective perception of the quality of urban areas is consistent with the concept of 15-minute cities, which purports that mixed-use areas with a diversity of amenities within a 15-minute walking distance are more attractive areas to live in---and by extension, to visit---than single-use ones. \citep{khavarian-garmsirGardenCity15Minute2023}

\section{Transport accessibility as attractors}

If the density and diversity of amenities were the only explanatory factors for non-commute trip demand, it would be tempting to conclude that having them alone is the sole determinant of why people go to one place and not another for their diverse needs. In reality, neighbourhoods do not exist in a vacuum but are intricately connected to one another by the transportation network. In simple terms, the consideration of distance as friction in the classic Gravity Model attests to the fact that the more accessible a destination is, the more likely it is to attract travel demand, especially trips with higher elasticity of choice, such as non-commute. Public transport accessibility has been shown to contribute to the growth of intercity importance of certain areas as opposed to others and preserve travel demand variability over time \citep{zhongMeasuringVariabilityMobility2015}, which means that it reinforces the prominence of well-connected areas as urban attractors. 

\citet{jayasingheApplicationDevelopingCountries2017} went one step further to purport that the network centrality property of an urban area is enough to predict travel demand to a high degree of accuracy without the need to consider other attractors such as employment centres and amenities, thanks to its high correlation. More specifically, the study looked at four centrality measures: Connectivity, Choice, Global Integration and Local Integration, which are analogous to the concept of degree centrality, betweenness centrality, closeness centrality, and closeness centrality in a confined radius, respectively. Among them, Global Integration (closeness centrality) was found to be the most significant predictor, suggesting that the role of the transport network in shaping urban travel demand at the aggregate scale is paramount, perhaps just as important as socio-economic factors \citep{converyDeterminantsTransportMode2019}. However, this study's use of car trip data does not represent all mobility patterns, especially considering public transport and walking. Therefore, it is logical to investigate the role of network centrality in conjunction with other variables.

\section{Public transport and the walkable city}

Walkability contributes to the quality of life in cities, with the "15-minute city" being one of the most prominent urban planning concepts that garnered attention in recent years. A resident living in a neighbourhood with all the amenities they need within a 15-minute walk or cycle can ideally reduce their reliance on cars and public transport to a minimum. However, this does not mean that the role of public transport is diminished within the 15-minute city vision. On the contrary, public transport is an essential component of a sustainable system, which provides an efficient and affordable way for people to travel longer distances in the city, effectively agglomerating a larger urban system made up of many 15-minute cities. In other words, public transport is the backbone of the 15-minute city, connecting neighbourhoods and enabling people to travel further than they can on foot or by bike. This, in turn, has multiple implications for the local communities:

\begin{itemize}
    \setlength\itemsep{0em}
    \item From the lens of urban planning and traffic management, the higher the volume of public transport trips to an area, the higher the amount of pedestrian traffic it receives on top of the local population patronising their local amenities. Understanding the trip attraction factors can help prioritise investments in public transport capacity planning, pedestrian or cycling infrastructures and services for these individuals, leading to a virtuous cycle of increased active travel and reduced car dependency.
    \item From the lens of fomenting local economies, increased pedestrian traffic injected by public transport can also be seen as an additional source of footfall, invaluable for businesses with storefronts. Findings on the economic benefits of walking and cycling done by Transport for London\footnote{https://tfl.gov.uk/corporate/publications-and-reports/economic-benefits-of-walking-and-cycling} showed that retail rents increased by 7\%, and office space rents increased by 4\% in areas that encouraged non-motorised access, suggesting that the street improvements translated into much more desirable spaces.
\end{itemize}

It is also worth mentioning that footfall estimation is a prolific field of research within retail analytics, which counts on various technologies such as WiFi tracking, video analytics, and mobile phone data to estimate the number of people passing by a given location. The insights derived from footfall data can inform decisions on store location, opening hours, staffing levels, and marketing strategies, among others. Uncovering public transport trip attractiveness patterns, which indicate non-local pedestrian inflows, could complement traditional footfall data and provide a more comprehensive retail footfall estimate.

\section{Explainable Artificial Intelligence for spatial systems}

Up to this point, our visited literature has shown that both urban amenities and transport accessibility are important factors in determining an area's attractiveness to non-commute trips, albeit in isolation. In reality, the interaction between these two factors has yet to be well studied in conjunction. In the interest of addressing the research questions, we are presented with a dilemma of interpretability versus prediction accuracy in the choice of modelling techniques.

Spatial statistical methods, commonly used in policy analysis, can help us quantify and interpret the relationship between the dependent variables and independent variables in the form of explicit coefficients. However, even spatially aware methods such as Spatial Lag Model (SLM) or Geographically Weighted Regression (GWR) are limited in their assumptions of linearity and issues with multicollinearity \citep{wheelerMulticollinearityCorrelationLocal2005}, to provide robust interpretations.

In parallel, machine learning methods have gained popularity in urban mobility research due to their ability to capture complex, non-linear relationships between spatial features with high prediction accuracy. However, the lack of interpretability of machine learning models has been a significant drawback, especially in urban planning, where understanding the underlying factors that drive the model's predictions is crucial for decision-making. Among many opportunities to develop Machine Learning explanation techniques for causal inference in urban mobility is the need to evaluate causal effects of input features and derive actionable insights \citep{xinVisionPaperCausal2022}.

To address the interpretability challenge, the Machine Learning interpretation technique like \textbf{SHAP} (SHapley Additive exPlanations) is a unified approach to explain the output of any machine learning model---introduced by the seminal work of \citet{lundbergUnifiedApproachInterpreting2017a} based on the cooperative game theory of Lloyd Shapley---, and aim to provide detailed attributions of input features to individual predictions made. Its introduction kick-started a growing body of literature on interpretable machine learning with spatial systems for regression with XGBoost \citep{liExtractingSpatialEffects2022} or classification with Light GBM \citep{louhichiShapleyValuesExplaining2023}. One of the most notable recent examples of SHAP for urban mobility in action was the development of the Deep Gravity Model introduced by \cite{siminiDeepGravityModel2021}, which used SHAP to not only explain how the predictions of the deep learning model for spatial interaction were influenced by the input spatial features such as land-use mix, points of interest in the origin and destination zones. 

As we approach this research with an eye on modelling the real-world phenomenon and explaining the contributing factors, instead of GWR or SLM, we will use a machine learning model to predict non-commute trip attraction in urban areas to a high degree of accuracy and then leverage SHAP to explain the predictions in an attempt to quantify the marginal importance and contribution of amenities and transport connectivity as attractors to an area's trip attractiveness.

\section{Open data and reproducibility}

Many recent advances in urban mobility research have been made possible by the availability of proprietary datasets owned by private companies, such as mobile phone data, GPS data, and social media data. Although the high data granularity has opened doors to new research opportunities and the development of new advanced techniques, its use also raises concerns about data privacy, data ownership, and data sharing. Moreover, it heightens the barrier to entry for independent researchers who would like to replicate the studies for their cities or regions. In contrast, open data not only democratises independent and participatory research in the public interest but also breaks down many reproducibility barriers when applying established methodologies to other localities \citep{yadavRoleOpenData2017}.

On one side, we have readily available open datasets maintained by local or federal government agencies in many countries. Conversely, we have global platforms like \textbf{OpenStreetMap} and \textbf{Overture Maps Foundation} that provide a wealth of spatial data for cities worldwide. The use of open data to encourage reproducibility will be a crucial focus of this research to ensure that the findings can be replicated, unburdened by commercial data availability and access barriers. It also enables flexible preprocessing of the amenity or transportation features to model the intracity mobility behaviour pertinent to the locality's needs while adhering to the same model training and analysis methodology. Adhering to the spirit of openness and future reproducibility across different urban contexts, all of the datasets used in this research will be entirely sourced from open data hubs and platforms. The associated scripts used for data processing, model training and prediction analytics are available on GitHub.

\section{Summary}

To address the research questions posed and informed by past literature, we set out to fine-tune and select a performant model to predict non-commute trip attractiveness, where we hypothesise that the built environment factors, including urban amenities and transport connectivity, are important trip attractors. Using a machine learning model in conjunction with explainable machine learning techniques such as SHAP allows us to capture complex, non-linear relationships and analyse contributions of input features to the predictions made to uncover underlying mobility patterns.