Urban mobility, a dynamic interplay of individual travel choices and collective movement patterns, shapes the fabric of cities. Critical aspects of urban mobility research and practical interests include a deeper understanding of why people make a trip, where they go, and how they get there, representing trip purposes, destination choice, and mode choice, respectively. Understanding, predicting, and analysing these patterns is not merely an academic pursuit but fundamental to sustainable urban planning and efficient urban transport service management. 

By closely examining the dimensions of trip purposes, we can distinguish between two broad categories of travel: commute and non-commute travel. Commute travel, which involves regular trips to and from work or school, has been the focus of much research and policy attention due to its predictable nature and significant impact on transport systems. On the other hand, non-commute travel encompasses a wide range of activities such as shopping, leisure, and socialising. It is more sporadic and diverse, making it challenging to model and predict, despite its demand share, especially when flexible working hours and remote working arrangements are becoming more common. Unlike commute travel, which is often well-researched and predictable to a large extent due to the regularity of work and school schedules, non-commute travel is more sporadic and diverse, encompassing a wide range of activities such as shopping, leisure, and socialising with no apparent fixed patterns. Even with abundant mobility data---varying in form from smart transit card taps to mobile location data from which origin-destination matrices could be extracted---the task of classifying and decomposing it into actionable insights remains challenging. 

Historically, analyses of non-commute travel have relied on costly and time-consuming travel surveys or proprietary methodologies and datasets. However, the increasing availability of open data sources---one of the most notable being OpenStreetMap (OSM)---and open-sourced machine learning techniques have introduced new possibilities for a broader segment of researchers and practitioners. Upon this context, we put forth the hypothesis that for non-commute travel, the built environment---as represented by the availability of amenities and the connectivity of transport networks---plays a significant role in influencing travel choices and patterns and can be used to predict this type of demand with the help of machine learning on open publicly-accessible data. However, we do not stop at predictions. While capable of producing high-accuracy predictions, the 'black box' nature of many machine learning models often obscures the decision-making process that could yield nuanced underlying factors that drive this demand. For this purpose, Explainable Machine Learning techniques, which belong to a growing body of literature on Explainable AI (XAI) research, can be positioned to provide a peek into the 'black box' and uncover the demand attractors in a quantifiable and interpretable manner. 

Using the Greater London region (United Kingdom) as the case study and tapping into a rich open dataset of amenity points of interest and transport networks extracted from OpenStreetMap, the UK Census 2021 and open public transport demand data from the municipal transport authority, Transport for London (TfL), this work will attempt to answer two key research questions:

\begin{enumerate}
    \item \textbf{Can non-commute trip attractiveness of an area be predicted using only amenity and transport connectivity features?} This question challenges us to refine our understanding of the relationship between the built environment and human mobility. Can these factors explain the variances in non-commute travel demand across different urban areas? Can we create accurate predictive models based solely on readily available open data, or are they inherently limited in their predictive power?
    \item \textbf{Can we identify destination hotspots/urban activity centres using mobility data?} This question builds upon our previous one. It explores the potential of explainable machine learning to quantify the spatially varying underlying factors that influence the model's prediction of non-commute travel across different urban areas. In doing so, we seek clarity into the spatial distribution of activity centres and amenity or transit hotspots.
\end{enumerate}

As previously purported, this research's findings can inform various urban planning and transport management decisions. By identifying the key factors that influence non-commute travel demand, we can help urban planners and policymakers design more efficient and sustainable transport systems. However, beyond use within the academic and planner communities, the insights generated by this research can also potentially interest communities and their businesses. For example, by understanding the physical factors that drive people to visit certain areas for specific purposes, footfall insights can be derived with high accuracy and used as part of the urban design process to improve the vibrancy of urban areas or in retail analytics to inform business decisions such as store locations and marketing strategies, etc. 

The dissertation is structured as follows: Chapter 2 provides a review of relevant literature on non-commute urban mobility where built environment features act as trip attractors and explainable machine learning techniques in spatial analytics. Chapter 3 describes the data preprocessing steps used in this research to engineer the target variables and features from open data, as well as the methodology used to train, evaluate and explain the prediction model to answer the research questions. Chapter 4 presents the results, followed by Chapter 5, which discusses the implications of the findings for urban planning and transport management, the limitations, and areas for future work. Finally, Chapter 6 concludes the dissertation by summarising this research's key findings and contributions.