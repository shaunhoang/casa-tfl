Urban mobility, a dynamic interplay of individual travel choices and collective movement patterns, profoundly shapes the fabric of modern cities. This complex phenomenon, influenced by a myriad of factors ranging from individual preferences and socioeconomic conditions to the characteristics of the built environment, serves as the lifeblood of urban areas. Understanding and predicting these intricate patterns is not merely an academic pursuit; it is fundamental to achieving sustainable urban planning and efficient transport system management.

Historically, insights into urban mobility have been gleaned through surveys and limited datasets, often characterized by their costliness and time-consuming nature. However, the advent of the digital age has ushered in a transformative era, one where open data, freely available and increasingly rich in detail, presents an invaluable resource for understanding urban dynamics. When harnessed in conjunction with advancements in machine learning, these open datasets can be transformed into powerful predictive models, shedding light on the intricate relationship between the built environment and human mobility.

This thesis embarks on a journey into this exciting frontier, seeking to predict non-commute transport demand using open data and machine learning techniques. In the face of rapid urbanization, cities worldwide are grappling with the challenges of managing escalating transport demand while ensuring the accessibility and sustainability of their transport systems. Urban planners and transport authorities require sophisticated tools to not only understand but also anticipate mobility patterns, empowering them to make informed decisions regarding infrastructure investments, service planning, and policy interventions. The confluence of open data and machine learning offers a promising avenue to address this pressing need.

At the core of this thesis lies the exploration of how features of the built environment, such as the availability of amenities and the connectivity of transport networks, influence the choices individuals make about their travel destinations and modes of transport. By constructing predictive models based on these features, we aim to forecast non-commute transport demand, providing valuable insights for urban planning and transport management.

However, we do not stop at mere prediction. We recognize that understanding the 'why' behind these predictions is equally crucial. The 'black box' nature of many machine learning models, while capable of producing accurate predictions, often obscures the underlying decision-making process. This lack of transparency can hinder their adoption in critical domains such as urban planning. To address this challenge, we employ Explainable AI (XAI) techniques. By peering into the 'black box', we strive to not only enhance trust in our models but also uncover valuable knowledge about the key drivers of non-commute transport demand.

In essence, this thesis serves as a testament to the transformative power of open data and machine learning in the realm of urban mobility research. We demonstrate how these tools can be leveraged to predict and understand non-commute transport demand, offering invaluable insights for urban planners and transport authorities. By shedding light on the complex relationship between the built environment and human mobility, we contribute to the creation of more sustainable, efficient, and livable cities.

However, we acknowledge that this research, like any other, has its limitations. The models we develop, while powerful, are inevitably simplifications of a complex reality. There is always room for improvement, whether through the incorporation of more granular data, the exploration of novel methodologies, or the expansion of our understanding of the factors influencing urban mobility.

In conclusion, we believe this thesis represents a significant step forward in harnessing the potential of open data and machine learning for urban mobility research. By demonstrating the feasibility of predicting non-commute transport demand and providing insights into the underlying factors, we pave the way for more informed and effective urban planning and transport management. As we look to the future, we are excited by the possibilities that lie ahead, and we are confident that data-driven approaches will play an increasingly central role in shaping the cities of tomorrow.

In the spirit of continued exploration and discovery, we leave you with two key research questions that have emerged from our work:

\begin{itemize}
    \item \textbf{Can trip attractiveness (total arrivals in an area) be predicted using only amenity and transport connectivity features?} This question challenges us to refine our understanding of the relationship between the built environment and human mobility. Can we create accurate predictive models based solely on readily available open data, or do we need to incorporate more complex socio-economic factors to achieve a comprehensive understanding?
    \item \textbf{Can we effectively identify destination hotspots/urban activity centers using mobility data such as total arrival data?} This question invites us to explore the potential of mobility data for revealing the hidden patterns of urban life. Can we pinpoint the areas that attract the most people, and what can this tell us about the dynamics and vibrancy of our cities?
\end{itemize}

These questions, and many others that arise from this research, point towards a future where data-driven insights play an increasingly central role in shaping our cities. We are confident that the work presented in this thesis represents a significant step towards that future, and we invite others to join us in this exciting journey of exploration and discovery.