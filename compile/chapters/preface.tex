\chapter*{Abstract}
Using SHAP explainable machine learning techniques on the predictions of an XGBoost model, we were able to use amenity and connectivity features extracted from open data of an area to predict with high accuracy ($R^2>80\%$) total arrivals by public transport in a given area in Greater London, representing the area's 'trip attractiveness'. The use of SHAP explanations further provides further insights into the spatial heterogeneity of the feature importance, enabling the identification of destination hotspots for different purposes, such as urban activities or transit interchange. The methodology can be extended to analysing public transport mobility patterns in other localities since it relies on open data sources such as OpenStreetMap and the national censuses. Moreover, it serves as a foundation to be developed and adapted to analyse intracity trip destination patterns from mode-agnostic mobility data to inform urban planning and policy-making decisions.

\section*{}
\textbf{Keywords:} urban mobility, explainable machine learning, XGBoost, SHAP, open data, activity hotspots, point-of-interests (POI), Greater London, public transport

\chapter*{Declaration}
I, The-Huan Hoang, declare that the work presented in this dissertation is my own. Where claims, information or findings have been derived from other authors and sources, I confirm that this has herein been indicated in writing, attributing due credit. I also acknowledge the use of Gemini Advanced LLM (Google, \url{https://gemini.google.com/}) for research note summarisation and draft proof-reading. Lastly, in the spirit of reproducible research, all scripts used for analysis and their outputs are made available on GitHub at \url{https://github.com/shaunhoang/casa-tfl}. 

\chapter*{Acknowledgements}
I am grateful for the guidance and insightful feedback from my supervisor, Prof. Elsa Arcaute, who has helped shape the direction of my work from its inception. I am also thankful for the invaluable support from the Public Transport Service Planning team at Transport for London (TfL). Their expert opinions, guidance around using TfL data, and real-world insights into the data science behind transport planning have greatly enriched my knowledge, as well as this research project. Last but not least, I would like to thank my family and friends, old and new, for their unwavering support, encouragement and inspiration as I took a leap of faith into navigating the brand new field of Urban Spatial Analytics. Without them, this journey would not have been as fulfilling and rewarding as it has been.

\tableofcontents

\listoffigures

\listoftables