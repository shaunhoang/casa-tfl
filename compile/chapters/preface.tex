\chapter*{Abstract}
Using SHAP explainable machine learning techniques on the predictions of an XGBoost model, we were able to use amenity and connectivity features extracted from open data of an area in Greater London to predict with high accuracy ($R^2>80\%$) total arrivals by public transport, representing the area's "trip attraction". The use of SHAP explanations further provided further insights into the spatial heterogeneity of the feature importance, enabling the identification of destination hotspots for different purposes, such as urban activities or transit interchange. The methodology can be extended to analysing public transport mobility patterns in other localities since it relies on open data sources such as OpenStreetMap and the national censuses. Moreover, it serves as a foundation to be developed and adapted to analyse intracity travel patterns from mode-agnostic mobility data to inform urban planning and policy-making decisions.

\section*{}
\textbf{Keywords:} urban mobility, explainable machine learning, XGBoost, SHAP, open data, activity hotspots, point-of-interests (POI), Greater London, public transport

\chapter*{Declaration}
I, The-Huan Hoang, declare that the work presented in this dissertation is my own. Where claims, information or findings have been derived from other authors and sources, I confirm that this has herein been indicated in writing, attributing due credit. I also acknowledge the use of Gemini Advanced (Google, \url{https://gemini.google.com/}) for research note summarisation and draft proofreading. Lastly, in the spirit of reproducible research, all scripts used for analysis and their outputs are made available on GitHub at \url{https://github.com/shaunhoang/casa-tfl}. 

\chapter*{Acknowledgements}
I am grateful for the guidance and insightful feedback from my supervisor, Prof. Elsa Arcaute, who has helped shape the direction of my work from its inception. I am also thankful for the invaluable support from the Public Transport Service Planning team at Transport for London (TfL). Their expert opinions, guidance and real-world insights into the data science behind transport planning have greatly enriched my knowledge and this research. Most importantly, I would like to thank my family, partner, and friends---old and new---for their unwavering support and encouragement as I took a leap into this challenging but fulfilling field of Urban Spatial Analytics. Lastly, I owe much to London for being the incredible backdrop to this journey. This research is a token of my appreciation for the city that has inspired me to continue learning and growing.

\tableofcontents

\listoffigures

\listoftables